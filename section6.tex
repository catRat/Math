\section{推理规则}
\subsection{引言}
本章后一部分我们将学习证明。数学中的证明是建立数学命题真实性的有效论证。所谓的 论证(argument),是指一连串的命题并以结论为最后的命题。所谓的 有效性 (valid),是指结论或论证的最后一个命题必须根据论证过程前面的命题或 前提(premise)的真实性推出。也就是说,一个论证是有效的当且仅当不可能出现所有前提为真而结论为假的情况。为从已知命题中推出新的命题,我们应用推理规则,这是构造有效论证的模板。推理规则是建立命题真实性的基本工具。
\subsection{命题逻辑的有效论证}

定义1  命题逻辑中的一个论证是一连串的命题。除了论证中最后一个命题外都叫作前提,最后那个命题叫结论。一个论证是有效的,如果它的所有前提为真蕴涵着结论为真。

命题逻辑中的论证形式是一连串涉及命题变元的复合命题。无论用什么特定命题替换其中的命题变元,如果前提均为真,则称为该论证形式是有效的。

\subsection{命题逻辑的推理规则}
我们可以先建立一些相对简单的论证形式(称为 推理规则)的有效性。

