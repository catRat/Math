\section{基础:逻辑与证明}
逻辑规则给出数学语句的准确含义。例如,这些规则有助于我们理解下列语句机器推理:“存在一个整数,它不是两个整数的平方和”,以及“对每个正整数 $n$ ,小于等与 $n$ 的正整数之和是 $n(n+1)/2$”。逻辑是所有数学推理的 基础,也是所有自动推理的基础。对计算机的设计、系统规范说明、人工智能、计算机程序设计、程序设计语言以及计算机科学的其他许多研究领域,逻辑都有实际应用。

为理解数学,我们必须理解正确的数学论证是由什么组成的。只要证明一个数学语句是真的,我们就称之为一个定理。关于一个主题的定理的集合就组成我们对这个主题的认知。为了学习一个数学主题,同我们需要积极地构关于此主题的数学论证,而不仅仅是阅读论证。此外,了解一个定理的证明通常就有可能通过细小的改动来获得适应新情况的结论。

每个人都知道证明在数学中的重要性,大许多人对于证明在计算机科学中的重要程度感到惊讶。事实上,证明常常用于验证计算机陈翔对所有可能的输入值产生正确输出值,用于解释算法总是产生正确结果,用于建立一个系统的安全性,以及用于创建人工智能系统。而且,自动推理系统已经被创造出来,让计算机自己来构造证明。
\section{命题逻辑}
\subsection{引言}
逻辑规则给出数学语句的准确含义,这些规则用来区分有效和无效的数学论证。由于本书的一个主要目的是教会读者如何理解和构造正确的数学论证,所以我们从介绍逻辑开始离散数学的学习。  

逻辑不仅对理解数数学推理十分重要,而且在计算机科学中有许多应用。这些逻辑规则用于计算机电路设计、计算机程序的构造、程序正确性证明以及许多其他方面。而且,已经开发了一些软件系统用于自动构造某些类型的证明。在随后的几章中将逐一讨论这些应用。

\subsection{命题}
逻辑的基本组件——命题。命题 是一个陈述语句,它或真或假,但不能即真又假。  

我们用字母来表示命题变元,它是表示命题的变量。习惯上用字母 
$p,q,r,s,\cdots$ 
表示命题。如果一个命题是真命题,它的真值为真,用$T$表示;如果它是假命题,其真值为假,用$F$来表示。  

涉及命题的逻辑领域称为命题演算或命题逻辑。它最初是2300多年前由古希腊哲学家亚里士多德系统地创建的。  

现在我们转而关注从已有命题产生新命题的方法。这些方法由英国数学家布尔在他的题为《The Laws of Thought》的书中讨论过。去多数学陈述都是由一个或多个命题组合而来。称为复合命题的新命题是由已知命题用逻辑运算符组合而来。  

定义1  令$p$为一命题,则$p$的否定记作 $\neg p$,指
\[
	“\mbox{不是p所指的情形}”
\]
命题 $\neg p$读作 “非$p$”。$p$的否定( $\neg p$)的真值和$P$的真值相反。  

例3  找出命题
“Michael的PC运行Linux。” 
的否定,并用中文表示。

下表是命题 $p$ 及其否定的 真值表 。此表列出命题 $p$ 的两种可能真值。每一行显示对应于 $p$ 的真值时 $\neg p$ 的真值。  

\begin{table}[h]
	\centering
	\topcaption{命题之否定的真值表}
	\begin{tabular}{l|l}
		\hline
		$p$ & $\neg p$ \\
		\hline
		T & F	\\
		F & T \\
		\hline
	\end{tabular}
\end{table}
		

命题的否定也可以看做否定运算符作用在命题上的结果。否定运算符从一个已知命题构造出一个新命题。现在我们将引入从两个或多个已知命题构造新命题的逻辑运算符。这些逻辑运算符也称为联结词。

定义2  令$p$和$q$为命题。
$p$、$q$
的合取即命题 “$p\mbox{并且q}$” ,记作 
$p \land q$ 
。当 $p$ 和 $q$ 都是真时,
$p \land q$
命题为真,否则为假。  

注意在逻辑合取中,有时候用“但是”一词。比如,语句“阳光灿烂,但是在下雨”是“阳光灿烂并且下雨”一句的另一种说法。  

例5  找出命题 $p$ 和 $q$ 的合取,其中 $p$ 命题“Rebecca 的 PC 至少有 16 GB空闲磁盘空间”,$q$ 为命题“Rebecca 的 PC 处理器的速度大于 1 GHz”。  

定义3  令$p$和$q$为命题。$p和q$的析取式即命题 “$p\mbox{或}q$”,记作$p \lor q$ 。当 $p\mbox{和}q$ 均为假时,析取式 $p\lor q$才为假,否则为真。  

在析取中使用的联结词或( or )对应于或在自然语言中所使用的两种情况之一,即兼或( inclusive or )。  
另一方面,当我们说
 “学过微积分或学过计算机科学,但不是两者都学过的学生,可以选修本课程”
的时候使用的是异或 (exclusive or)。这里的意思是既学过微积分,又学过计算机科学的学生不能选修本课程。只有那些恰好在这两门课中修过一门的学生可以选修本课程。

定义4  令 $p和q$ 为命题。$p\mbox{和}q$ 的异或是这样一个命题:当 $p\mbox{和}q$ 中恰好只有一个为真时命题为真,否则为假。

\subsection{条件语句}
下面讨论其他几个重要的命题合成方式。

定义5  令 $p和q$ 为命题.条件语句 $p \to q$ 是命题“如果$p$,则$q$”。当 $p$ 为真而 $q$ 为假时,条件语句 $p \to q$ 为假,否则为真。在条件语句 $p \to q$ 中,$p$ 称为假设,$q$ 称为结论.

由于条件语句在数学推理中具有重要的作用,所以表达 $p \to q$ 的术语也 很多,即使不是全部,你也会碰到下面几个常见的条件语句的表述方式:

 "$\mbox{如果}p,\mbox{则}q$"

 "$p\mbox{蕴含}q$"

 "$\mbox{如果}p,q$"

 "$p\mbox{仅当}q$"
 
 "$p\mbox{是}q\mbox{的充分条件}$"

 "$q\mbox{的充分条件是}p$"

 "$q\mbox{如果}p$"

 "$q\mbox{每当}q$"

 "$q\mbox{当}p$"

 "$q\mbox{是}p\mbox{的必要条件}$"

 "$p\mbox{的必要条件是}q$"

 "$q\mbox{由}p\mbox{得出}$"

 "$q\mbox{除非}\neg p$"

条件语句 $p \to q$ 的众多表达方式中有两个最容易引起混淆的是:“ $p\mbox{仅当}q$ ”和“ $q \mbox{除非} p$ ”。因此,这里提供一些消除混淆的建议。

请记住“ $p\mbox{仅当}q$ ”表达了与"$\mbox{如果}p,\mbox{则}q$"同样的以,注意“ $p\mbox{仅当}q$ ”说的是当 $q$ 不为真时 $p$ 不能为真。要小心不用用 “$q \mbox{仅当} p$”来表达 $p \to q$,因为这是错误的。要明白这一点,请注意当 $p$ 和 $q$ 取不同的真值时,“ $q\mbox{仅当}p$ ”和 $p \to q$的真值是不同的。


例7 令 $p$ 为语句 ''Maria学习离散数学'', $q$ 为语句 ''Maria会找到好工作''。用中文表达语句 $p \to q$ 。

''如果 Maria 学习离散数学,那么她会找到好工作''

''当 Maria 学习了离散数学,她会找到月份好工作''

''Maria 会找到一份好工作,她只要学习了离散数学就足够了''

''Maria 会找到一份好工作,除非她不学习离散数学''


许多程序设计语言总使用的 if-then 结构与逻辑中使用的不同。大部分程序设计语言中都有 if $p$ then $S$ 这样的语句,其中 $p$ 是命题而 $S$ 是一段程序段。当程序的运行遇到这样一条语句时,如果 $p$ 为真,就执行 $S$;如果 $p$ 为假,则 $S$ 不执行。  

逆命题、逆否命题与反命题 由条件语句 $p \to q$ 可以构成一些新的条件语句。特别是三个常见的相关条件语句还拥有特殊的名称。命题 $q \to p$ 称为 $p \to q$的 逆命题,而 $p \to q$ 的 逆否命题 是命题 $\neg q \to \neg p$ 。命题 $\neg p \to \neg q$ 称为 反命题 。我们发现,三个由 $p \to q$ 衍生的条件语句中,只有逆否命题总是和 $p \to q$ 具有相同的真值。  

当两个复合命题总是具有相同的真值,我们称它们是等价的。  

双条件语句  下面我们介绍另外一种命题复合方式来表达两个命题具有相同的真值。  

定义6  令 $p$ 和 $q$ 为命题。双条件语句 $p \to q$ 是命题 "$p\mbox{当且仅当}q$"。当 $p$ 和 $q$ 有同样的真值时,双条件语句为真,否则为假。双条件语句也称为双向蕴含。

''$p\mbox{是}q$的充分必要条件''

''如果 $p$ 那么 $q$,反之依然''

''$p$ 当且仅当 $q$''。


例10  令 $p$ 语句“你可以搭乘该航班”,令 $q$ 为语句“你买票了”。则 $p \leftrightarrow q$ 为语句
“你可以搭乘该航班当且仅当你买机票了”

此语句为真,如果 $p$ 和 $q$ 均为真或均为假。  

双条件的隐式使用  你应该意识到在自然语言中双条件并不总是显式地使用。特别是在自然语言中很少使用双条件中的“当且仅当”结构。通常用“如果,那么”或“仅当”结构来表示蕴含。“当且仅当”的另一部分是隐含的。也就是逆命题是蕴含的而没有说明出来。例如,考虑这个自然语言的语句“如果你吃晚饭了,则可以吃餐后甜点”。其真正含义是“你可以吃餐后甜点当且仅当你吃完饭”。后而这个语句在逻辑上等价于两个语句“如果你吃晚饭了,那么你可以吃甜点”和“仅当你吃完了饭,你才能吃甜点”。由于自然语言的这种不精确性,所以我们需要对自然语言中的条件语句是否蕴含它的逆做出假设。因为数学和逻辑注重精确,所以我们总是区分条件语句 $p \to q$ 和双条件语句 $p \leftrightarrow q$。

\subsection{复合命题的真值表}

我们已经介绍了否定以及4个重要的逻辑联结词——合取、析取、条件、双条件。我们可以用这些联结词来构造含有一些命题变元的季候复杂的复合命题。我们可以用真值表来决定这些复合命题的真值。

\subsection{逻辑运算符的优先级}

现在,我们可以用所定义的否定运算符和逻辑运算符来构造复合命题。我们通常使用括号来规定复合命题中的逻辑运算符的作用顺序。例如,$(p \lor q) \land (\neg r)$ 是 $p \lor q$ 和 $\neg r$ 的合取。然而,为了减少括号的数量,我们规定否定运算符先于所有其他运算符。这意味着 $\neg p \land q$ 是 $\neg p$ 和 $q$ 的合取,即 $(\neg p) \land q$ ,而不是 $p$ 和 $q$ 的合取的否定,即 $\neg(p \land q)$。   
另一个长用的优先级规定则是合取运算符优先于析取运算符,因此 $p \land q \lor r$ 意味着 $(p \land q) \lor r$ ,而不是 $p \land (q \lor r)$ 。因为这个规则不太好记,所以我们将继续使用括号是析取运算符和合取运算符的作用顺序看起来很明了。  

最后,一个已被接受的规则是条件运算符和双条件运算符的优先级低于合取和析取运算符。因此, $p \lor q \to r$ 等同于 $(p \lor q) \to r$ 。当设计条件运算符和双条件运算符的作用顺序时,我们也将使用括号,尽管条件运算的优先级高于双条件运算。

\begin{table}
	\begin{tabular}{l|l}
		\hline
		运算符 & 优先级 \\
		\hline
		$\neg$	& 1 \\
		$\land \quad \lor$ & 2 3 \\
		$\to \quad \leftrightarrow$ & 4 5 \\
		\hline
	\end{tabular}
\end{table}


\subsection{逻辑运算和位运算}

计算机用位表示信息。位是一个具有两个可能值的符号,即0和1。位一词的含义来自二进制数字(binary digit),因为 0 和 1 是数的二进制表示中用到的数字。1946 年著名的统计学家约翰-图基引入了这一术语。一位可以用于表示真值,因为只有两个真值,即真和假。习惯上,我们用 1 表示真, 用 0 表示假。即,1 表示 T , 0 表示 F 。 如果一个变量的值或为真或为假,则此变量称为 布尔变量。于是一个布尔变量可以用一位表示。

\subsection{习题}
\begin{enumerate}
	\item	
		下列那些语句是命题?这些是命题的语句的真值是什么?
		\begin{enumerate}
			\item	南宁是广西的首府
			\item	2+3=5
			\item	$5+7=10$
			\item	$x+2=11$
			\item	回答这个问题
		\end{enumerate}
	\item
		下列个命题的否命题是什么?
		\begin{enumerate}
			\item	Mei有一台MP3播放器
			\item	右江区没有污染
			\item	$2+1=3$
			\item	右江区的夏天又热又晒
		\end{enumerate}
	\item
		下列各命题的否定是什么?
		\begin{enumerate}
			\item	Steve的笔记本电脑有大于100GB的空闲磁盘空间。
			\item	Zach阻止来自Jennifer的邮件和短信。
			\item	$7 \cdot 11 \cdot 13 = 999 $
			\item	Diane周日骑自行车骑了200公里。
		\end{enumerate}
	\item
		假设在最近的财年期间,Acme计算机公司的年收入是1380亿美元且其净利润是80亿美元,Nadir软件公司的年收入是870亿美元且将利润是50亿美元,Quixote媒体的年收入是1110亿美元且净利润是130亿美元。试判断有关最近财年的每个命题的真值。
		\begin{enumerate}
			\item	Quixote媒体的年收入最多。
			\item	Nadir软件公司的净利润最少且Acme计算机公司的年收入最多。
			\item	Acme计算机公司的净利润最多或者Quixote媒体的净利润最多。
			\item	如果Quixote媒体的净利润最少,则Acme计算机公司的年收入最多。
			\item	Nadir软件公司的净利润最少当且仅当Acme计算机公司的年收入最多。
		\end{enumerate}
	\item
		令
		$p$
		和
		$q$
		分布表示命题“在新泽西海岸游泳是允许的”和“在海岸附近发现过鲨鱼”。试用汉语表达下列每个复合命题。
		\begin{enumerate}
			\item	$\neg q$
			\item	$p \land q$
			\item	$\neg p \lor q$
			\item	$p \to \neg q$
			\item	$\neg q \to p$
			\item	$\neg p \to \neg q$
			\item	$p \neg q$
			\item	$\neg p \land (p \lor \neg q)$
		\end{enumerate}
	\item
		令
		$p$
		、
		$q$
		为如下命题:
		$p$:气温在零度以下
		$q$:正在下雪。
		用
		$p$
		、
		$q$
		和逻辑联结词写出下列各命题:
		\begin{enumerate}
			\item	气温在零度以下且正在下雪。
			\item	气温在零度以下,但没有下雪。
			\item	气温不在零度以下,清切没有下雪。
			\item	也许正在下着雪,也许在零度以下。
			\item	如果气温在零度以下,则下着雪。
			\item	也许气温字零度以下,也许下着雪;但如果在零度以下,就没有下雪。
			\item	气温在零度以下是下雪的充分必要条件。
		\end{enumerate}
	\item
		令
		$p$
		、
		$q$
		为如下命题:

		$p$:你的车速超过每小时104公里。

		$q$:你街道一张超速罚单。

		用$p$、$q$和逻辑联结词写出下列命题:
		\begin{enumerate}
			\item	你的车速没有超过每小时104公里。
			\item	你的车速超过每小时104公里,但没有街道超速罚单。
			\item	如果你的车速超过每小时104公里,你将接到一张超速罚单。
			\item	如果你的车速不超过每小时104公里,你就不会街道超速罚单。
			\item	车速超过每小时104公里足以街道超速罚单。
			\item	只要你接到一张超速罚单,你的车速就超过每小时104公里。
		\end{enumerate}
	\item
		令
		$p$
		、
		$q$
		、
		$r$
		为如下命题:
		$p:$在这个地区发现过灰熊。

		$q:$在详见小路上徒步旅行是安全的。

		$r:$详见小路两旁的草莓成熟了。

		用pqr和逻辑联结词写出下列命题:
	\item
		判断下列个条件语句是真是假:
		\begin{enumerate}
			\item	如果$1+1=2$,则$2+2=5$。
			\item	如果$1+1=3$,则$2+2=4$。
			\item	如果$1+1=3$,则$2+2=5$。
			\item	如果猴子会飞,那么$1+1=3$。
		\end{enumerate}
	\item
		下列各语句,判断其中想表达的是兼或还是异或,说明理由。
		\begin{enumerate}
			\item	晚餐有咖啡或者茶。
			\item	口令必须至少包含3个数字或至少8个字符长。
			\item	这门课程的先修课程是数论课程或者密码课程。
			\item	你可以用没有或者欧元支付。
		\end{enumerate}
	\item
		对下列各语句,说一说如果其中的联结词是兼或与异或时的含义。你认为语句想表达的是哪个或?
		\begin{enumerate}
			\item	要选修离散数学课,你必须已经选修了微积分或一门计算机的课程。
			\item	当你从Acme汽车公司购买一部新车时,你就能得到2000美元先进折扣或2\%的汽车贷款。
			\item	两人套餐包括A栏中的两道菜或B栏中的三道菜。
			\item	如果下雪超过0.6米或寒风指数低于-100,学校就停课。
		\end{enumerate}
	\item
		把下列语句写成“如果$p$,那么$q$”的形式。
		\begin{enumerate}
			\item	只要吹东北风,就会下雪。
			\item	苹果树会开发,如果天暖持续一周。
			\item	活塞队赢得冠军就意味着他们打败了湖人队。
			\item	必须走8英里才能到达郎思峰的登峰。
			\item	想要得到终身教授职位,只要能世界闻名就够了。
			\item	如果你驾车超过400英里,就需要买汽油了。
			\item	你的保修单是有效的,只有当你购买的CD机不超过90天。
			\item	Jan要去游泳,除非水太凉了。
		\end{enumerate
		\end{enumerate}
\end{enumerate}
