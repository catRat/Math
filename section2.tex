\section{ 命题逻辑的应用}
\subsection{引言}
逻辑在数学、计算机科学和其他许多学科有着许多重要的应用。数学、自然科学以及自然语言中的语句通常不太准确,甚至有歧义。为了使用精确表达,可以将他们翻译成逻辑语言。例如,逻辑可用于软件和硬件规范(specification)描述,因为在开发前这些规范必须要准确。另外,命题逻辑及其规则可用于设计计算机电路、构造计算机程序、验证程序的正确性以及构造专家系统。逻辑可用于分析和求解许多熟悉的谜题。基于逻辑规则的软件系统也已经开发出来,它能够自动构造每种类型证明。在后续章节中,我们将讨论命题逻辑的部分应用。
\subsection{语句翻译}
有许多理由要把语句翻译成由命题和逻辑联结词组成的表达式。特别是,汉语(以及其他各种人类语言)常用二义性。把语句翻译成复合命题可以消除二义性。注意,这样翻译时也许需要根据语句的含义做一些合理的假设。此外,一旦完成了从语句到逻辑表达式的翻译,我们就可以分析这些逻辑表达式以确定他们的真值,对他们进行操作,并用推理规则对它们进行推理。  

为了解释把语句翻译成逻辑表达式的过程,考虑下面两个例子。  
例1  怎样把下面的语句翻译成逻辑表达式? “你可以在校园访问因特网,仅当你主修计算机科学或者你不是新生”

解  我们的办法是用命题变元表示语句中的每个成分,并找出他们之间合适的逻辑联结词。上述语句可以翻译为
\[
a \to ( c \lor \neg f)
\]
\subsection{系统规范说明}
在描述样件系统和软件系统时,将自然语言语句翻译成逻辑表达式是很重要的一部分。系统和软件工程师根据子软语言描述的需求,生成精确而无二义性的系统规范说明,这些规范说明可作为系统开发的基础。例3说明了如何在这一过程中使用复合命题。  
例3  使用逻辑联结词表示规范说明“当文件系统已满时,不能够发送自动应答”。  

解  翻译这个规范说明的方法之一是令 $p$ 表示“能够发送自动应答”,令 $q$ 表示“文件系统满了”,则 $\neg p$ 表示“并非能够发送自动应答这种情况”,也就是“不能够发送自动应答”。因此我们的规范说明可以用条件语句 $q \to \neg p$ 来表示。 

系统规范说明应该是 一致的 ,也就是说,系统规范说明不应该包含可能导致矛盾的相互冲突的需求。当规范说明不一致时,就无法开发出一个满足说有规范说明的系统。

例4 确定下列系统规范说明是否是一致的。

“诊断消息存储在缓冲区中或者被重传。”

“诊断消息没有存储在缓冲区中。”

“如果诊断消息存储在缓冲区中,那么它被重传”
\subsection{布尔搜索}
\subsection{逻辑谜题}
\subsection{逻辑电路}
\subsection{习题}
在联系1~3中,用给定的命题将语句翻译成命题逻辑中的形式。
\begin{enumerate}
	\item	你不能便器一个受保护的维基百科条目,除非你是一名管理员。用$e:$“你不能标记一个受保护的维基百科条目”和$a:$“你是一名管理员”来表达你的答案。
	\item	你能够毕业仅当你已经完成了专业的要求并且你不欠大学学费并且你没有预期不归还图书馆的书。用$g:$“你能够毕业”,$m:$“你不欠大学学费”,$r:$“你已经完成了专业的要求”和$b:$“你没有语气不归还图书馆的书”来表达你的答案。
	\item	你有之歌当美国总统仅当你已年满35岁、出生在美国或者你出生时你的双亲是美国公民并且你在这个国家至少生活了14年。用$e:$“你有之歌当美国总统”,$a:$“你已经棉麻你35岁”,$b:$“你出生在美国”,$p:$“在你出生的时候,你的双亲均是美国公民”和$r:$“你在美国至少生活了14年”来表达你的答案。
\end{enumerate}
	
